\section{Introduction} \label{sec1}
The negative consequences of the HIV virus range from individual impediments and the risk of death, to macroeconomic effects in detriment of economic growth. Since the discovery of the AIDS virus 35 years ago, several agencies started to work towards a common objective: complete eradication of the virus.\\
  
The challenge of erradication is larger in developing countries where the effectiveness of policy intervention is conditioned by financial constraints. However this is not the only hurtle. Governments struggle because they have limited understanding of the effects of policy intervention across the HIV epidemic. In particular the \cite{report2} report, states that intervention should be based on specific epidemiology facts, such as the stage of the epidemic and those groups at high risk. This is recommended because policy efficacy might as well vary over the evolution of the epidemic. \\

Targeted prevention suggests that irrespectively of the stage of the epidemic an efficient way to reduce HIV diffusion is to target the groups at high risk. As described by the \cite{report2} report, certain groups have a higher chance to get and spread HIV. These groups include most of the time individuals engaging in high-risk sexual behavior such as sex workers and their clients, injecting drug users and recently, men who have sex with men. How about individuals across educational groups?\\

\cite{beegle} argue that there is a lack of consensus on the sign and the size of the relationship between education and HIV exposure, in other words, it is still not clear if people with lower education have a higher risk of infection than educated individuals. \cite{raul} provides empirical evidence on the relationship between HIV status and education. He finds that risky behavior that increases HIV exposure, differs across educational groups as the HIV epidemic evolves. In particular, the education gradient in HIV has a U shape across the HIV epidemic. This means that in the early stages of the epidemic, additional years of education are associated with an increase of the probability of being HIV infected. This fact is the main motivation for this paper.   \\

In this paper I propose a general equilibrium model that studies extramarital risky sex and its implications for HIV diffusion. The model treats the market of extramarital risky sex as any other competitive market, in which agents pay a competitive price and consume an amount of the desired good. In the model, agents differ principally in their education level and their level of asset holdings. Agents can be of two types: sex buyers or sex producers. The model has the feature that the probability of getting infected depends on the amount of extramarital risky sex an individual consumes. That is, the infection probability is endogenously determined by the model. In principle, agents are infinitely lived, however they have a probability of dying. Additionally, population grows at a constant rate.\\

The model divides the HIV epidemic in four stages, each with different features that intend to capture the particularities found in the data. In the first stage (\textit{pre-epidemic stage}) there is no infection risk. Infection starts in stage two ((\textit{myopic stage})) where infected individuals suffer a productivity loss and therefore are payed less for their labor. Moreover, infected individuals have a higher probability of dying. However, in this stage agents are not aware of the risk of infection and they don't take into account the new characteristics of the environment to make their consumption choices. In other words, agents are myopic because they might be infected and observe the productivity loss, but they don't know the reason for this loss and they ignore it. In stage three (\textit{maturity stage}), agents update their beliefs and take into account the probability of infection in their consumption decisions. Finally in stage four (\textit{ARTs stage}), antiretroviral treatment is provided for all infected individuals. Introduction of antiretroviral drugs represent a gain in productivity, a reduction of the probability of infection and an increase in the survival probability of infected individuals.\\

I propose an algorithm that joins all four stages of the model with the intention to characterize the complete evolution of the HIV epidemic. Each stage after the first arrives as an unpredictable shock. The algorithm intends to capture the behavior of the agents in each stage, in particular the feature that during the \textit{myopic stage} and \textit{ARTs stage}, agents with higher education might have a higher exposure to HIV infection because of their decision to engage in more extramarital risky sex\footnote{However this might not be the case of Malawi}. However, once they realize the negative effects of the disease and learn about the risk of infection, they decide to reduce their consumption of extramarital risky sex, and therefore reducing their probability of infection. In the \textit{ARTs stage}, agents regain productivity levels even if they are infected, this in turn reverts the results of the \textit{maturity stage} by making educated agents increase their risky sexual activities and therefore increasing HIV exposure. Further simulations of the model can be conducted to study the transitions between epidemic stages, however these simulations are left for future research.\\

Heterogeneous agent models are very rich in the sense that they generate data even without performing any type of simulation. I exploit this feature of the model to extract the associated prevalences and match them to the different stages of the Malawian epidemic. It is in this way that the model is calibrated for Malawi. The data the model provides includes the prevalences across stages of the different types of agents and levels of education, proportion of infected individuals among the educated people and proportion of infected sex buyers and sex producers, relative to the total population. I use this information to run a linear probability model to explain HIV exposure using as explanatory variables, the level of education of the population and its interactions with each stage of the epidemic. In addition, I provide a disaggregation by gender to better compare the results to the data. This requires an additional assumption regarding the gender of the individuals. For simplicity reasons I assigned to men, the role of sex buyers and to women, the role of sex producers, however there is no reason that restricts contrary.\\

The results are not surprising as they are in line with the empirical results. Malawi doesn't have a clear U shaped educational gradient at the aggregate level, however the U shape is clearly visible for women. During the \textit{myopic stage}of the HIV epidemic finishing secondary education increases the probability of infection by 2.9${\%}$ among the total population. Among males and females 4.41${\%}$ and 2.42 ${\%}$ respectively. During the \textit{maturity} of the epidemic finishing secondary education reduces the chanced of infection by 1.45$\%$; $0.6\%$ for men and $1.8\%$ for females. Finally during the \textit{ARTs stage} education actually reduces the risk of infection by  3.16${\%}$ among the total population and among men and women 3.39${\%}$ and 0.79 ${\%}$ respectively. 

The rest of the document is organized as follows, Section\ref{sec2} summarizes the state of the art regarding HIV diffusion and education. Section\ref{sec3} describes the evolution of the Malawian HIV/AIDS epidemic. Section\ref{sec4} makes a detail description of the proposed model. Section\ref{algorithm} presents the solution algorithm, finally Section\ref{sec6} explores the results of the model and the calculation of the education gradient for Malawi, Section\ref{sec7} concludes. Further resources are provided in the Appendix\ref{appendix}

\section{Relevant literature}\label{sec2}

The literature on HIV exposure and education is very small. In fact, there is limited research that propose a theoretical framework to study the spread of HIV/AIDS, the use of general equilibrium models to asses implications of the HIV epidemic are on its infancy \cite{michelle}. The present study intends to fills this gap.\\
An exeption to the previous claim is the work by \cite{yaoyao}. She constructs a life cycle model that relates sex decisions, fertility and education to HIV risk. Females choose between committed sex and casual sex. Committed sex tends to result in more premarital children because premarital relations and committed sex facilitate marriage. When HIV risk increases, women choose to reduce premarital children and choose committed sex, since premarital sex and casual sex are positively related to HIV infection. Women with higher levels of education tend to have less committed sex and are more likely to choose casual sex, since having children has a higher opportunity cost for educated women. However, less educated women choose committed sex since they have a lower opportunity cost.  The author proposes three policies to mitigate HIV exposure and concludes that an education subsidy is the most effective in reducing HIV prevalence, due to the reduction of premarital risky sex practices and the increase of the opportunity cost of child bearing.\\


\cite{fort} also argues that education has a positive relation with premarital risky sex that increases the probability of HIV infection. Using data from the Demographic and Health Survey (DHS) he finds evidence of a positive education gradient in HIV infection. This means that high levels of education are more likely to get HIV infected. However this applies to individuals with very high levels of education, namely six years of schooling or more. \cite{mirsha} find a link between household wealth and HIV status, they document that in Sub Saharan Africa the wealthiest quintiles have a higher prevalence of HIV, and prevalence increases monotonically with wealth. This positive association can partly be lined to other factors such as geographical position of the household and the level of education. \\

Other authors use a different approach, and find opossite results. \cite{paxton} study the role of education on the origins and geographic concentration of the HIV epidemic. He argues that the regions where HIV grew quickly, had an increase of non-marital sex activity and growing female education. Other autors like \cite{preston} and \cite{bledsoe} argue that education only reduces the chances of HIV infection if the health risks are completely understood. Additionally, education might increase the chance of poligamous partnerships and reduce early mariage that in turn increases the chance of infection.\cite{oster} tracks back the problem to an important increase of risky sexual behavior among educated women. \cite{brent} used a cross-section of 31 Sub Saharan countries to examine the effect of the level of education of females on the prevalence of HIV/AIDS, they find that female education is positively related with the infection rates. However when disentangling between primary and secondary education and accounting for different types of education within them \footnote{for example consider all the age groups in the sector(gross) or  as an alternative consider only those who are at the official school age (net)}, their results show that education is negatively related to female HIV infection if the variable \textit{"females on the official school age"} is used instead of the alternative.\\

Following a similar argument, \cite{walque} finds a negative relation between education and HIV diffusion. The author uses individual level data on the effectiveness of an information campaign  to prevent HIV/AIDS in Uganda. They find that education is associated with a lower HIV infection risk among educated young females.  They claim that educated women are more responsive to information campaigns. In the same line \cite{alsan} find that girls enrollment in secondary education significantly increased sex abstinence, this in turn coincides with the dramatic fall of HIV prevalence in Uganda in the early 1990's.\\

In an attempt to reach a generalizable answer to the mixed evidence about the effect of education on HIV difussion,\cite{raul} constructs an algorithm that characterizes the stages of the HIV epidemic across time and space, and computes the HIV education gradient for a panel of Sub Saharan countries. The author shows that the education gradient in HIV has a U shape over the course of the epidemic. Moreover men and women share the same U pattern but women have more pronounced shape. Specifically he concludes that in early stages, women with five additional years of education face a 7.4 $\%$ rise in the probability of infection. This in contrast to 3.8$\%$ in the case of men. In the middle of the epidemic this numbers reduce to zero for women and negative for men. In advanced stages the gradient becomes positive again with 2.7$\%$ for women and 1.75$\%$ for men\footnote{Refer to the Appendix\ref{appendix} for a visual presentation of the educational gradient found by \cite{raul}}. This dynamic relation between the education level of the population, HIV prevalence and the stage of the epidemic is the starting point for the present research.\\

A richer model was proposed by\cite{michelle}, where they analyze the Malawian HIV epidemic through the lenses of a choice theoretic general equilibrium search model. In their model, people choose among a set of sexual practices by being aware of the associated risk. They argue that the effectiveness of intervention depends also on a behavioral effect that is capable of having adverse macro effects by increasing HIV prevalence. This behavioral effect involves, engaging in more risky sexual practices, and lower incentive of protection. \cite{kremer}formally models partner choices in  an attempt to introduce behavioral considerations into epidemiology, he argues that HIV risk is associated with the people's sexual activity choices. A higher risk causes individuals with a low sexual activity to reduce it even more, but high activity people reduce their activity less than the low activity ones, or even increase it. This differences might result in an less favorable equilibrium for the economy.\\

\cite{manuelli} analyzes the effects of HIV/AIDS on aggregate output by studying the influence of the disease on investment directed to education and job training. He builds a human capital accumulation model, where individuals accumulate human capital in the form of schooling and job training. The accumulation decisions are influenced by the disease, since an infected individual might face a higher probability of permanent disability or death which in turn can significantly limit the individuals maximum level of effort, and at the macro level, cause a reduction of output per-capita.\\

\section{Country Profile Malawi}\label{sec3}
\begin{center}
\textit{"Malawi has one of the highest HIV prevalences in the world despite the impressive progress the country has made in controlling its HIV epidemic in recent years."\cite{report1}}
\end{center}

Since the first case of HIV-AIDS diagnosed in Malawi, the country has suffered from a rapid increase of the number of people infected with the virus. Figure\ref{bild1} in the Appendix\footnote{For presentation purposes, all of the figures in this section have been taken to the Appendix\ref{figures}} shows the evolution of the Malawian HIV prevalence and Figure\ref{bild2} shows the evolution of the number of people living with HIV, new infections and mortality due to HIV. We can see from Figure\ref{bild1}that the Malawian epidemic reaches its peach in year 1999 where the prevalence rate reached $16.7$ percent of the population. After the peak the prevalence has been constantly decreasing, nevertheless still high compared to other countries. However, when looking at Figure\ref{bild2} we can see that the number of people living with HIV increases drastically until the year 2000, but after that it keeps growing until today.  Although the share of infected people is reducing, the increase of the total number of infected could not be mitigated. On the other hand, the number of new infections of HIV/AIDS has been constantly reducing and it has reached its lowest point in 2016. We can also see that the number of deaths due to HIV/AIDS reduced since 1999, however it did not reach its lower value of 1990.\\    


To better understand the Malawian epidemic it is important to dig deeper as transmission rates vary depending on gender, age and other socioeconomic variables such as the level of education, aspect that has not been deeply studied. Transmission happens mostly due to unprotected heterosexual sex between either cohabiting or married couples. In fact, premarital sex is common in Malawi to encourage marriage, although this is considered as a risky practice. Infection also happens due to mother to child transmission, and men who have sex with men.  \\

Figure\ref{bild4} shows the share of females as a percentage of the total population living with HIV. It is alarming to see that more than half of the population who is infected are females. Moreover, this share has been constantly increasing, but since 2016 the rate started to grow faster reaching 60 percent of the infected population in 2016. Figure\ref{bild3} the number of AIDS related deaths by gender. Since the start of the epidemic the number of infected children increased by a factor of $3.5$ up to 2003, it was only in 2004 that this growth stopped, and only in 2011 it started to reduce. A similar pattern can be spotted when looking that the number of deaths related to HIV, as only after 2004 the number of deaths started to reduce and almost reached pre-epidemic levels by 2016. It is important to note that women infected with HIV have a higher mortality rate than men until 2011, after that year, the situation reversed.\\     

UNAIDS identified: women, young people engaged in early sexual activity, sex workers, gay men and other men who have sex with men as the most vulnerable sectors of the population in Malawi. But what about people with different education levels?. As strange as it may sound, it is not the less educated individuals who have a higher HIV prevalence, but exactly the opposite. Table\ref{table1} shows the prevalence rate dis-aggregated by education level. Table\ref{table2} shows the educational attainment in Malawi and its disaggregation by gender. We can see that the amount of people that finished secondary school is very small, in turn there is a very small number of females that finished high school. Although the situation slightly improved, Malawi still has a long way to go to increase educational attainment.\\

\begin{table}[H]
\centering
\caption{Malawi, HIV prevalence among the general population (in $\%$)}
\label{table1}
\begin{tabular}{c|c|c|c}
\hline
\multirow{2}{*}{\textbf{}}&\multirow{2}{*}{\textbf{Total}}&\multicolumn{2}{|c}{\textbf{By education level}}\\
\cline{3-4}
 &   &  \textbf{Primary or less}&  \textbf{Secondary or more}\\
 \hline \hline
 DHS 2004& 11.8 & 11.1 & 14.0 \\
 [0.3em]
 DHS 2010    & 10.7 & 10.2 & 11.8 \\
 [0.3em]
\hline \hline
\end{tabular}
\begin{flushleft}
Sources: The prevalences have been taken form DHS data.
\end{flushleft}
\end{table}

\begin{table}[H]
\centering
\caption{Malawi, Population over age 6 who completed secondary education(in $\%$)}
\label{table2}
\begin{tabular}{c|c|c|c}
\hline
\multirow{2}{*}{\textbf{Survey}}&\multicolumn{3}{|c}{\textbf{Share of (*) with secondary education or higher}}\\
\cline{2-4}
 &   \textbf{(*)Total population}&  \textbf{(*)Females}&  \textbf{(Males)}\\
 \hline \hline
 DHS 1992& 9.4 & 4.4 & 14.3 \\
 [0.3em]
 DHS 2000& 16.0 & 11.1 & 20.9 \\
  [0.3em]
 DHS 2004& 21.4 & 15.5 & 27.2 \\
  [0.3em]
 DHS 2010& 25.6 & 20.0 & 31.2 \\
  [0.3em]
 DHS 2015& 31.2 & 25.8 & 36.5 \\
\hline \hline
\end{tabular}
\begin{flushleft}
Sources: The prevalences have been taken form DHS data.
\end{flushleft}
\end{table}

Table\ref{table1} shows that the overall HIV prevalence in Malawi reduced from  $11.8$ percent in 2004 to $10.7$ in 2010, the same occurs when we see the disaggregation by educational attainment. The prevalence of the people with primary or less education decreased slightly, keep in mind that this is a period in which the country expected the largest reduction in HIV prevalence at the national level. In the other hand, the prevalence among individuals with secondary education and higher, decreased more. However, when comparing the prevalence between less educated individuals and people with a higher educational attainment, we can see that every year, those who are more educated face a larger risk of getting infected. The gap reduced subtancialally in six years, nevertheless it still remains. \cite{raul} documents this fact for 
a number of Sub Saharian countries and attributes the cause to the fact that educated individuals believe they are les vulnerable therefore they increase their risky sexual practices.\cite{michelle} studies the possibility of endogenous behavioral macro effects that play an important role when determining the effectiveness of intervention. They claim that up until a point these behavioral macro effects might reverse the positive effects of treatment. \\

The Malawian government, in cooperation  with various international donors (UNAIDS, USAID to name the most important) implemented intensive mitigation programs during the 90's. These included the promotion of male circumcision and the treatment of other sexually transmitted diseases, but the most important and until now the most effective was the introduction of antiretroviral therapy in 2005 . By 2012 almost forty percent of the population living with HIV received antiretroviral treatment and by 2016 the threshold surpassed fifty percent, the malawian governement seeks to increase the level of coverage to $90$ percent to achieve the UNAIDS 90,90,90\footnote{The 90,90,90 targets stand for: 90 percent of people living with HIV knowing their status, 90 percent of people living with HIV who know their status are on treatment and finally 90 percent of people on treatment are viraly supresssed} HIV prevention goals. Although the Malawian governement claimed the introduction of antiretroviral terapy was a complete success due to the aparent reduction of HIV prevalence, researchers suggest the effect is more ambiguous than expected \cite{michelle}. Despite all the efforts Malawi remains with one of the highest HIV prevalence rate in the world. \\
