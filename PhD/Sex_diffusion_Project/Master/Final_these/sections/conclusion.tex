\section{Conclusions}\label{sec7}
In this paper I propose a general equilibrium heterogeneous agent model intended to help understand the dynamic relation between education and HIV diffusion. The model features two types of agents that differ primarily in their level of education, asset holdings and HIV status. The model identifies four stages of the HIV-epidemic: 1.A \textit{pre-epidemic stage}, where there is no risk of infection. 2. A \textit{myopic stage} where individuals do not realize the hazard of HIV infection and don't take it into account for consumption decisions. 3.A \textit{maturity stage} where individuals acknowledge the risk of infection and immediately adjust their expectations and consumption decisions. It is because of this reason that the equilibrium amount of extramarital risky sex in the maturity of the epidemic is less than that the \textit{myopic stage}. Finally the \textit{ARTs stage} introduces antiretroviral treatment that might cause the prevalence among educated people to increase. This can be explained by the fact that since educated individuals believe they have a lower risk of infection, then they start to increase their levels of extramarital risky sex again and this has overall positive effect on prevalences. \\

 The paper proposes an innovative algorithm that links all four epidemic stages together in order to have a dynamic representation of the HIV epidemic. The model is then calibrated for Malawi using mostly DHS data. Once the model is solved, I use the data generated by the model to calculate the HIV education gradient for Malawi and its dis aggregation by gender.\\
 
 The gradient has an evident U shape, however, after the \textit{myopic stage} the gradient becomes significantly negative, meaning that it is recommendable to increase the level of education of the population in order to reduce the risk of HIV infection. Moreover when looking at the gender dis-aggregation we note that education indeed reduces the probability of infection both  among men and women. But for the case of women after the \textit{maturity of the epidemic}, the education gradient starts to get closer to zero. This suggests that increasing the level of education of the female population in Malawi will not reduce HIV prevalence as in the \textit{maturity of the epidemic}. Quantitatively speaking, during the \textit{myopic stage}of the epidemic additional education increases the probability of infection by 2.9${\%}$ among the total population and among males and females 4.41${\%}$ and 2.42 ${\%}$ respectively. However during the \textit{ARTs stage} education actually reduces the risk of infection by  3.16${\%}$ among the total population and among men and women 3.39${\%}$ and 0.79 ${\%}$ respectively. 