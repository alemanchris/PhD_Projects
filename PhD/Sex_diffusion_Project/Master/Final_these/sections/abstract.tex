\vspace{2cm}
\begin{center}
\Large{\bf{Abstract}}\\
\end{center}
\indent In this paper I examine the role of education on the evolution of the exposure to HIV risk across the HIV epidemic. Cross country empirical results show that the education gradient in HIV is U-shaped. This means that at some point during the HIV epidemic, agents with higher education have a higher risk of becoming HIV positive. I propose a heterogeneous agent general equilibrium model to account for this stylized fact. Additionally, the paper proposes an algorithm to fully characterize the evolution of the HIV epidemic across stages.  The model is calibrated to match the prevalences of the Malawian HIV epidemic. Finally, using the data generated by the model I calculate the education gradient for Malawi, the results show that the shape of the education gradient is in line with the empirical evidence. The model paves the road further policy implications, for example targeted prevention depending on the stage of the HIV epidemic. 
%\footnote{The attached MATLAB code 'Laffer_curve_Bolivia.m' contains the solution of the model,the main results and graphs}.

