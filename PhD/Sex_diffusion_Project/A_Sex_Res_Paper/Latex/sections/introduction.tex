\section{Introduction}\label{sec1}

Targeted prevention suggests that irrespective of the stage of the epidemic an efficient way to reduce HIV diffusion is to target the groups at high risk. As described by the \cite{report2} report, certain groups have a higher chance to get and spread HIV. These groups include mostly individuals engaging in high-risk sexual behavior such as sex workers and their clients, injecting drug users and recently, men who have sex with men. How about individuals across educational groups?\\
\cite{beegle} argue that there is a lack of consensus on the sign and the size of the relationship between education and HIV exposure, in other words, it is still not clear if people with lower education have a higher risk of infection or if its the other way around. \cite{raul} provides empirical evidence on the relationship between HIV status and education. He finds that risky sex behavior (associated with high HIV exposure), differs across educational groups as the HIV epidemic evolves. In particular, the education gradient in HIV has a U shape across the HIV epidemic. This means that in the early stages of the epidemic, additional years of education are associated with an increase of the probability of being HIV infected. This fact is the main motivation for this paper.   \\
This paper proposes a general equilibrium model that studies extramarital risky sex and its implications for HIV diffusion. The model treats the market of extramarital risky sex as any other competitive market, in which agents pay a competitive price and consume an amount of the desired good. In the model, agents are heterogeneous and differ principally in their education level and their level of asset holdings. Agents can be of two types: sex buyers or sex producers. In the model, the probability of HIV infection depends on the amount of extramarital risky sex an individual consumes. That is, the infection probability is endogenously determined by the model. In principle, agents are infinitely lived, however they have a random  probability of dying. Additionally, population grows at a constant rate.\\